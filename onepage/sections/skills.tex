\cvsection{Skills}
{
    \normalsize

\textbf{Programming}: C/C++, Python, Java, Haskell, Bash Scripting, Verilog\\
\textbf{Web}: PHP, HTML5, CSS, Javascript, SQL, NodeJS\\
\textbf{Utilities}: MPI, OpenMP, CUDA, IntelTBB, Linux Shell Utilities, Git, \LaTeX, Vim\\
}
\vspace{2mm}
%%% Local Variables:
%%% mode: latex
%%% End: