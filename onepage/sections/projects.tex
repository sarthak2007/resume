\cvsection{Projects}
\begin{cventries}
{\normalsize
  
  \cventry
  {Course Project, Operating Systems}
  {\href{https://github.com/sarthak2007/OperatingSystems_CS330_Assignments}{Building GemOS}}
  {}
  {July 2019 - Nov 2019}
  {
    \begin{cvitems}
      \item Implemented file system syscalls including open, write, pipe, etc.
      \item Implemented multi-level paging management for syscalls like mmap, munmap and mprotect.
      \item Implemented smart process creation for system calls like cfork and vfork by properly managing the processes.
      \item Designed a read-write lock and implemented multi-threaded hashtable with Open Addressing using mutual exclusion
      devices like locks and semaphores for preventing concurrent access.
      \vspace{2mm}
    \end{cvitems}
  }

  \cventry
  {Assignments, Programming for Performance}
  {\href{https://github.com/sarthak2007/Programming-for-performance}{Parallel Programming}}
  {}
  {Oct 2020 - Nov 2020}
  {
    \begin{cvitems}
      \item Implemented program optimizations such as loop transformations, vectorization using Intel SSE/AVX Intrinsics for achieving tremendous speedups (10x-30x) in serial programs.
      \item Used Intel TBB \& OpenMP and wrote optimized CUDA kernels for extacting performance benefits from programs such as Prefix sums, 3D stencil computations and Quicksort.
      \vspace{2mm}
    \end{cvitems}
  }

  \cventry
  {Course Project, Principles of Programming Languages}
  {\href{https://github.com/sarthak2007/Oz_Interpreter}{Oz Interpreter}}
  {}
  {Nov 2020}
  {
    \begin{cvitems}
      \item Developed an interpreter from scratch for a simple kernel language, Oz.
      \item Implemented all the basic features of a declarative sequential language such as application of non-suspendable and suspendable statements, unification of variables and values, maintenance of a single assignment store and a semantic stack, and pattern matching.
      \vspace{2mm}
    \end{cvitems}
  }
  
  \cventry
  {Course Project, Machine Learning}
  {DCaptcha}
  {}
  {Nov 2019}
  {
    \begin{cvitems}
      \item Built a CAPTCHA decoder using OpenCV for Image Preprocessing and Segmentation.
      \item Used PyTorch for building a CNN for character recognition after segmentation.
      \item Got accuracy of 100\% on test dataset after training the model on a dataset of 2000 images of similar styled captcha.
      \vspace{2mm}
    \end{cvitems}
  }
}
\end{cventries}

%%% Local Variables:
%%% mode: latex
%%% TeX-engine: xetex
%%% TeX-master: "../cv.tex"
%%% End: