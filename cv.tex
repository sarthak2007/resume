%!TEX TS-program = xelatex
%!TEX encoding = UTF-8 Unicode
% Awesome CV LaTeX Template for CV/Resume
%
% This template has been downloaded from:
% https://github.com/posquit0/Awesome-CV
%
% Author:
% Claud D. Park <posquit0.bj@gmail.com>
% http://www.posquit0.com
%
% Template license:
% CC BY-SA 4.0 (https://creativecommons.org/licenses/by-sa/4.0/)
%


\documentclass[10pt, a4paper]{awesome-cv}
\geometry{left=1.4cm, top=.8cm, right=1.4cm, bottom=1.8cm, footskip=.5cm}
\fontdir[fonts/]

% Color for highlights
% Awesome Colors: awesome-emerald, awesome-skyblue, awesome-red, awesome-pink, awesome-orange
%                 awesome-nephritis, awesome-concrete, awesome-darknight
\colorlet{awesome}{awesome-darknight}
% Uncomment if you would like to specify your own color
% \definecolor{awesome}{HTML}{CA63A8}

% Colors for text
% Uncomment if you would like to specify your own color
% \definecolor{darktext}{HTML}{414141}
% \definecolor{text}{HTML}{333333}
% \definecolor{graytext}{HTML}{5D5D5D}
% \definecolor{lighttext}{HTML}{999999}

% Set false if you don't want to highlight section with awesome color
\setbool{acvSectionColorHighlight}{false}

% If you would like to change the social information separator from a pipe (|) to something else
\renewcommand{\acvHeaderSocialSep}{\quad\textbar\quad}


% Available options: circle|rectangle,edge/noedge,left/right
% \photo{./profile.png}
\name{Yash}{Srivastav}
\position{Senior Undergraduate{\enskip\cdotp\enskip}Computer Science and Engineering}
\address{Indian Institute of Technology, Kanpur}
\mobile{(+91) 705-413-3662}
\email{yash111998@gmail.com}
\homepage{yashsriv.org}
\github{yashsriv}
\linkedin{yashsriv}
% \twitter{@therealyashsriv}
% \quote{``There is no fate but what we make."}

\newcommand{\smallcventry}[6]{\cventry{#1}{#2}{#3}{#4}{#6}}
\newcommand{\specialcvsection}[1]{\cvsection{#1}}




\begin{document}
\makecvheader
\makecvfooter
  {}
  {}
  {\thepage}

\cvsection{Educational Qualifications}
\begin{cventries}
{\Large
  
  \cventryedu
  {Bachelor of Technology, Computer Science and Engineering}
  {}
  {}
  {July 2017 - May 2021(Expected)}
  {
    \begin{cvitems}
      \item Cumulative Performance Index/CPI: \textbf{9.7/10}
    \end{cvitems}
  }


}
\end{cventries}

%%% Local Variables:
%%% mode: latex
%%% TeX-engine: xetex
%%% TeX-master: "../cv.tex"
%%% End:
\cvsection{Scholastic and Programming Achievements}
\begin{cvhonorsa}
{\Large
  

  \cvhonora
  {Received Academic Excellence Award}
  {2017 - 18}
  {}
  {2017}
  
  \cvhonora
  {All India Rank \textbf{169}}
  {Joint Entrance Exam Advanced among 200,000 candidates}
  {}
  {2017}

  \cvhonora
  {All India Rank \textbf{78}}
  {Joint Entrance Exam Mains}
  {}
  {2017}
}

\end{cvhonorsa}
\begin{cvhonors}
{\Large
  \cvhonor
  {Qualified for \textbf{ACM ICPC} Amritapuri regionals 2018, Secured \textbf{173$^{rd}$} rank in regionals}
  {}
  {2018}
  
  \cvhonor
  {Ranked \textbf{31$^{st}$} among the Indian teams in Google Hash Code 2019}
  {}
  {2017}

  \cvhonor
  {Rated 5 star on CodeChef with rating of 2026}
  {}
  {2017}
  
  \cvhonor
  {Expert on Codeforces with maximum rating 1682}
  {}
  {2017}

}
\end{cvhonors}

\cvsection{Work Experience and Projects}
\begin{cventries}
{\Large
  
  \cventry
  {Boost C++ Organization}
  {Google Summer of Code}
  {}
  {May 2019 - Present}
  {
    \begin{cvitems}
      \item Integrated Boost.Units with the existing base coordinate system and restructured all the classes to make them compatible with units to provide a robust astronomical coordinate system.\vspace{1mm}
      \item Created the parser for binary table extension and ASCII table extension for FITS File system.
        \vspace{1.0mm}
      \item Used Template Meta Programming in C++ to provide almost no runtime overhead and allow users to write scientifically infallible code by detecting all the errors at the compile time.
     
      \vspace{1mm}
        \ifdefined \ONEPAGE \else
      \item Technologies Used: DBus, C++, CUPS, Google Cloud Printing
        \fi
    \end{cvitems}
  }

\cventry
  {Summer of Code, Prof. Sandeep Shukla}
  {Full Stack Developer Intern}
  {}
  {May - July 2018}
  {
    \begin{cvitems}
      \item Developed a dynamic and interactive web application from scratch as an initiative to improve the medical system by keeping track of
records of patients and their history.\vspace{1mm}
      \item Developed a question-answer platform for medical system using PHP and MYSQL in the backend with frontend made using HTML and Javascript.
        \vspace{1mm}
      \item Technologies and languages used: PHP, MYSQL, AJAX, HTML, Javascript, Microsoft Azure.
      \vspace{1mm}
      \item Project sponsored by NUTANIX and UPSIDC
      \vspace{1mm}
        \ifdefined \ONEPAGE \else
      \item Technologies Used: DBus, C++, CUPS, Google Cloud Printing
        \fi
    \end{cvitems}
  }
  
  \cventry
  {Association for Computing Activities, Department of CSE, IIT Kanpur}
  {Basics of Reinforcement Learning}
  {}
  {Jan 2018 - Mar 2018}
  {
    \begin{cvitems}
      \item Learned about Markov Decision Processes, Q-Learning and SARSA.\vspace{1mm}
      \item Implemented Q-learning algorithm to find the shortest possible path to reach diagonally opposite corner in a square grid with blocked paths of no prior information.
        \vspace{1mm}
        \ifdefined \ONEPAGE \else
      \item Technologies Used: DBus, C++, CUPS, Google Cloud Printing
        \fi
    \end{cvitems}
  }
}
\end{cventries}

%%% Local Variables:
%%% mode: latex
%%% TeX-engine: xetex
%%% TeX-master: "../cv.tex"
%%% End:
\cvsection{Skills}
{\Large

\textbf{Programming}: C/C++, Python, Haskell, Bash Scripting, Verilog\\
\vspace{1.5mm}
\textbf{Web}: PHP, HTML5, CSS, Javascript, MYSQL, NodeJS, React Native\\
\vspace{1.5mm}
\textbf{Utilities}: Linux Shell Utilities, Git, \LaTeX, Vim
\vspace{1.5mm}
}
\vspace{3mm}
%%% Local Variables:
%%% mode: latex
%%% End:
\cvsection{Relevant Courses}

\ifdefined\ONEPAGE

% \textbf{CS:} Introduction to Programming(A$*$), Logic in Computer
% Science, Computer Organization, Data Structures and Algorithms, Computing
% Laboratories - 1(A$*$)
{\Large
\vspace{2.5mm}
\begin{tabular*}{\textwidth}{l l l}
  Introduction to Programming & Linear Algebra & Introduction to Logic\\   Probability for Computer Science &
Data Structures and Algorithms & Discrete Mathematics\\
Software Labs & Computer Organization & Theory of Computation \\ Algorithms-II &
  Introduction to Machine Learning & Operating Systems \\
  Computer Architecture($i$) & Database Management($i$) & Compiler Design($i$)
\end{tabular*}



% \textbf{Math}: Discrete Math, Probability and Statistics(A$*$)


{\footnotesize
    {{\large ~~$i$: Upcoming}}
}

\else
{\fontsize{11pt}{1em}\bodyfontlight\upshape\color{text}
  \begin{tabular*}{\textwidth}{l l l}
    Introduction to Programming(A$*$) & Discrete Mathematics  & Computer Organization \\
    Computer Architecture & Data Structures and Algorithms & Probability \& Statistics(A$*$) \\ 
    Computing Laboratories - 1(A$*$) & Computing Laboratories - 2(A$*$) & Compiler Design \\
    Functional Programming(A$*$) & Computer Systems Security & Computer Networks($i$)
  \end{tabular*}
}
{\fontsize{11pt}{1em}\footerfont\upshape\color{text}
  \begin{tabular*}{\textwidth}{ l l }
    \entrylocationstyle{A$*$: Grade for exceptional performance} & \entrylocationstyle{$i$: In progress}\\
  \end{tabular*}
}

\fi
}
%%% Local Variables:
%%% mode: latex
%%% TeX-engine: xetex
%%% TeX-master: "../cv"
%%% End:
\newpage
\cvsection{Programming Achievements}
\begin{cvhonors}
{\Large
  \cvhonor
  {Qualified for \textbf{ACM ICPC} Amritapuri regionals 2018, Secured \textbf{173$^{rd}$} rank in regionals}
  {}
  {2018}
  
  \cvhonor
  {Ranked \textbf{31$^{st}$} among the Indian teams in Google Hash Code 2019}
  {}
  {2017}

  \cvhonor
  {Rated 5 star on CodeChef with rating of 2026}
  {}
  {2017}
  
  \cvhonor
  {Expert on Codeforces with maximum rating 1682}
  {}
  {2017}

}
\end{cvhonors}

%%% Local Variables:
%%% mode: latex
%%% TeX-engine: xetex
%%% TeX-master: "../cv"
%%% End:
\cvsection{Positions of Responsibility}

\begin{itemize}
\item \textbf{Head, Web}, \emph{Antaragni 2017}
  : \\
  Worked on a full MEAN stack application. As part of the Core Team was involved
  in decisions regarding the festival and was responsible for managing the stays
  and travels of all Celebrities and Artists invited to the event.
\item \textbf{Coordinator}, \emph{Programming Club, IIT Kanpur 2017-18}
  : \\
  Conducted lectures for freshmen and organised competitions. Took the
  initiative of conducting a \textbf{``Winter Camp''} where a select few
  freshmen were introduced to various topics.
  % ranging from cryptography to web development.
% \item \textbf{Secretary}, \emph{Programming Club, IIT Kanpur 2016-17}
%   \ifdefined\ONEPAGE
%   \else
%   : \\
%   Helped Conduct and organize various lectures for freshmen as well as developed
%   a few web applications under the programming club.
%   \fi
% \item \textbf{Senior Executive, Web}, \emph{Antaragni 2016}
%   \ifdefined\ONEPAGE
%   \else
%   : \\
%   Worked on a NodeJS webserver for a college fest. Had a dynamic website
%   modifiable easily by non-programmers and supported android app as well with an
%   API.
%   \fi
\end{itemize}

\cvsection{Miscellaneous}

\begin{itemize}
  % \item Developed a Python Application using \textbf{Pygame} 
  %   \ifdefined \ONEPAGE . \else
  %   for 2 player as well as
  %   single player Reversi gameplay as part of ACA Semester Project.
  %   Link -
  %   \href{https://github.com/yashsriv/Reversi-Python}{github://yashsriv/Reversi-Python}
  %   \fi
  %   \vspace{-1mm}
  % \item Ported the educational OS, nachos, to golang. Link -
  %   \href{https://github.com/yashsriv/go-nachos}{github:yashsriv/go-nachos}
  %   \vspace{-1mm}
  % \item Developed an AI for complete-knowledge two-player games in Haskell as a
  %   course project.
  %   \ifdefined \ONEPAGE \else
  %   Implemented Connect 4 with GUI as an instance of that AI.
  %   Link - \href{https://github.com/yashsriv/haskell-connect-4}{github://yashsriv/haskell-connect-4}
  %   \fi
  %   \vspace{-1mm}
  {\Large
  \item Mentored a project on algorithms and data structures including segment trees, sparse table, Kruskal's algorithm, Floyd-Warshall\vspace{1mm}
  \item Secretary of Association of Computing Activities, IIT Kanpur \vspace{1.5mm}
    \ifdefined \ONEPAGE \else
    using nodejs and
    websockets as an Semester Project
    \fi
    \vspace{2mm}
    }
\end{itemize}
% \cvsection{Interests}

{\fontsize{11pt}{1em}\bodyfontlight\upshape\color{text}
  \begin{itemize}
  \item Open Source
  \item Capture The Flag Contests
  \item Web Development
  \item Image Processing
  \item Artificial Intelligence
  \item Robotics
  \end{itemize}
}

%%% Local Variables:
%%% mode: latex
%%% TeX-engine: xetex
%%% TeX-master: "../cv"
%%% End:


\end{document}

%%% Local Variables:
%%% mode: latex
%%% TeX-engine: xetex
%%% End: